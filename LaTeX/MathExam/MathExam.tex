%-*- coding: UTF-8 -*-
% MathPaper.tex
%!TEX TS-program = xelatex
%!TEX encoding = UTF-8 Unicode
\documentclass[a4paper, 12pt]{article}
%设置中文字体
\usepackage{fontspec}
\setromanfont{STSongti-SC-Regular}
\XeTeXlinebreaklocale "zh"
\XeTeXlinebreakskip = 0pt plus 1pt minus 0.1pt %文章内中文自动换行

%版面设置x
\usepackage{geometry}
\geometry{left=2cm, right=2cm, top=1cm, bottom=2cm}

%首行缩进
\usepackage{indentfirst}
\setlength{\parindent}{2.45em}

%表格
\usepackage{booktabs}
%矩阵
\usepackage{amsmath}

%全局行间距
\linespread{1.6}

\title{高等数学结束习题}
\author{戚海军 \\学号: MP1533024 \\Email: 2316828815@qq.com}
\date{}

\begin{document}

\maketitle

\section{第一部分}
1. 设V是有序实数对的集合:\begin{math}V = \{(a, b) | a, b\in∈R\}\end{math}。规定其加法(\begin{math}\oplus\end{math})和数乘(\begin{math}\circ\end{math})运算为\begin{math}(a, b)\oplus(c, d) = (a, b)\end{math},与\begin{math}k\circ(a, b) = (ka, kb)\end{math}。请问V关于运算\begin{math}\oplus、\circ\end{math}是否构成R上的线性空间。
%矩阵论第二讲1

答:

要想V构成R上的线性空间,必须满足加法和数乘的所有性质。
	
\textbf{加法性质}:
对\begin{math}\forall x,y\in V,定义x+y\in V\end{math},且满足:
	
1. 交换律	\begin{math}x + y = y + x\end{math}
	
2. 结合律 \begin{math}x + (y + z) = (x + y) + z\end{math}
	
3. 存在唯一的零元素,记作0,且
\begin{math}x + 0 = x\end{math}
		
4. 加法的逆运算,记作
\begin{math}-x + (-x) \equiv x - x = 0\end{math}
		
\textbf{数乘性质}:
对\begin{math}\forall k\in R, x\in V\end{math},定义\begin{math}kx\end{math}满足:
	
5. \begin{math}kx\in V\end{math}
	
6. 结合率 \begin{math}k(ax) = (ka)x\end{math}
	
7. 分配律 \begin{math}k(x + y) = kx + ky; (a + b)x = ax + bx\end{math}
	
8. F中存在单位元(记成1),满足
\begin{math}1x = x\end{math}
		
验证:

\begin{math}(c, d)\oplus(a, b) = (c, d)\end{math}

\begin{math}(a, b)\oplus(c, d) = (a, b)\end{math}

得出\begin{math}(c, d)\oplus(a, b)\neq(a, b)\oplus(c, d)\end{math}

由此可见运算\begin{math}\oplus\end{math}不满足加法交换律,不能构成R上的线性空间。
\newpage
2. 假定\begin{math}\alpha_{1}, \alpha_{2}, \alpha_{3}\end{math}是\begin{math}R^{3}\end{math}的一组基,试求由\begin{math}\beta_{1} = \alpha_{1} - 2\alpha_{2} + 3\alpha_{3}, \beta_{2} = 2\alpha_{1} + 3\alpha_{2} + 2\alpha_{3}, \beta_{3} =4 \alpha_{1} + 13\alpha_{2}\end{math}生成的子空间\begin{math}Span(\beta_{1}, \beta_{2}, \beta_{3})\end{math}的基底。
%矩阵论第二讲1

答:

设\begin{math}k_{1}\beta_{1} + k_{2}\beta_{2} + k_{3}\beta_{3} = 0\end{math}
\[
\left\{                  %方程组的左边包括大括号\{
\begin{array}{lll}     %设定列阵的格式:{lll}是三个L,表示三列的对齐方式为Left对齐
k_{1} + 2k_{2} + 4k_{3} = 0\\
-2k_{1} + 3k_{2} + 13k_{3} = 0\\
3k_{1} + 2k_{2} = 0\\
\end{array}           %方程列阵的结束
\right.              %方程组的右边无符号,利用“.“来标示
\]
$$\begin{bmatrix}
1& 2& 4& 0\\
-2& 3& 13& 0\\
3& 2& 0& 0
\end{bmatrix}
\sim
\begin{bmatrix}
1& 2& 4& 0\\
0& 1& 3& 0\\
0& 0& 0& 0
\end{bmatrix}
$$

\[
\Rightarrow
\left\{                  %方程组的左边包括大括号\{
\begin{array}{lll}     %设定列阵的格式:{lll}是三个L,表示三列的对齐方式为Left对齐
k_{1} = 2k_{3}\\
k_{2} = -3k{_3}\\
\end{array}           %方程列阵的结束
\right.              %方程组的右边无符号,利用“.“来标示
\Rightarrow
2k_{1} - 3k_{2} + k_{3} = 0
\]

\begin{math}
dimSpan\{\beta_{1}, \beta_{2}, \beta_{3}\} = 2;
\end{math}

基底为\begin{math}\beta_{1}, \beta_{2}或\beta_{2}, \beta_{3}或\beta_{1}, \beta_{3}\end{math}。
\newpage
3. 设\begin{math}V_{1}\end{math}是内积空间\begin{math}V^{n}\end{math}的任一子空间,试证明存在唯一的子空间\begin{math}V_{1}^{\bot}\subset V^{n}\end{math}使得\begin{math}V_{1}\oplus V_{1}^{\bot} = V^{n}\end{math}。
%矩阵论第三讲 P40 定理2.11

证:

存在性:

设\begin{math}\alpha_{1}, \alpha_{2}, \ldots, \alpha_{m}(m < n)为V_{1}\end{math}的标准正交基,经扩充后\begin{math}\alpha_{1},\alpha_{2}, \ldots, \alpha_{m}, \alpha_{m+1}, \ldots, \alpha_{n}\end{math}为V的标准正交基。若取\begin{displaymath}V_{2} = Span(\alpha_{m+1}, \alpha_{m+2}, \ldots, \alpha_{n})\end{displaymath}显然\begin{math}V_{1}\perp V_{2}且有V^{n} = V_{1}\oplus V_{2} = V_{1}\oplus V_{1}^{\bot}\end{math}。

唯一性:

若另有\begin{math}V_{1}\end{math}的正交补空间\begin{math}V_{3}\end{math}使\begin{displaymath}V_{1}\oplus V_{3} = V^{n}\end{displaymath}则对任意\begin{math}0\ne\beta\in V_{3}\end{math},有\begin{math}\beta\notin V_{1}\end{math},且\begin{displaymath}(\alpha, \beta) = 0, \forall\alpha\in V_{1}\end{displaymath}所以\begin{math}\beta\in V_{2},即V_{3}\subset V_{2},同理可证V_{2}\subset V_{3},故有V_{2} = V_{3}\end{math}。

结论:存在唯一的子空间\begin{math}V_{1}^{\bot}\subset V^{n}\end{math}使得\begin{math}V_{1}\oplus V_{1}^{\bot} = V^{n}\end{math}。
\newpage
\section{第二部分}
以下数据是2010年度美国总统经济报告中的部分数据,其中反映了从2001到2007年五个国家的GDP(国内生产总值)的增长率。在经济全球化的形势下,欧美国家的经济相互依存,相互影响。

试根据这组数据利用最小二乘法,建立美国GDP增长率与德国、法国、意大利、西班牙四国经济增长率的线性模型。并假设某一年度受金融危机影响,这四个国家的经济增长率分别为1.2,0.3,-1.0和0.9时,预测美国的预期增长率是多少?
\\

\begin{tabular}{lccccccc}
\toprule
Country& 2001& 2002& 2003& 2004& 2005& 2006& 2007\\
\midrule
United States& 1.1& 1.8& 2.5& 3.6& 3.1& 2.7& 2.1\\
Germany& 1.2& 0& -0.2& 1.2& 0.7& 3.2& 2.5\\
France& 1.8& 1.1& 1.1& 2.3& 1.9& 2.4& 2.3\\
Italy& 1.8& 0.5& 0& 1.5& 0.7& 2.0& 1.6\\
Spain& 3.6& 2.7& 3.1& 3.3& 3.6& 4.0& 3.6\\
\bottomrule
\end{tabular}
\\
%最小二乘法
%http://zh.numberempire.com/matrixcalculator.php
%http://www.yunsuanzi.com/cgi-bin/least_squares.py

答:

设德国、法国、意大利和西班牙分别为\begin{math}x_{0}, x_{1}, x_{2}, x_{3}\end{math}。建立线性模型
\begin{displaymath}f(x_{0}, x_{1}, x_{2}, x_{3}) = ax_{0} + bx_{1} + cx_{2} + dx_{3}\end{displaymath}

把数据代入模型,得到以下不相容的线性方程组:
\[
\left\{                  %方程组的左边包括大括号\{
\begin{array}{lll}     %设定列阵的格式:{lll}是三个L,表示三列的对齐方式为Left对齐
1.2a + 1.8b + 1.8c + 3.6d = 1.1\\
1.1b + 0.5c + 2.7d = 1.8\\
-0.2a + 1.1b + 3.1d = 2.5\\
1.2a + 2.3b + 1.5c + 3.3d = 3.6\\
0.7a + 1.9b + 0.7c + 3.6d = 3.1\\
3.2a + 2.4b + 2.0c + 4.0d = 2.7\\
2.5a + 2.3b + 1.6c + 3.6d = 2.1
\end{array}           %方程列阵的结束
\right.              %方程组的右边无符号,利用“.“来标示
\]

相应的不相容线性方程组是\begin{math}Ax = b\end{math},其中:

$$A = \begin{bmatrix}
1.2& 1.8& 1.8& 3.6\\
0& 1.1& 0.5& 2.7\\
-0.2& 1.1& 0& 3.1\\
1.2& 2.3& 1.5& 3.3\\
0.7& 1.9& 0.7& 3.6\\
3.2& 2.4& 2.0& 4.0\\
2.5& 2.3& 1.6& 3.6
\end{bmatrix}, 
b = \begin{bmatrix}
1.1\\
1.8\\
2.5\\
3.6\\
3.1\\
2.7\\
2.1
\end{bmatrix}$$

正规方程\begin{math}A^{T}Ac = A^{T}b\end{math}

\[
\left[ {\begin{array}{cccc}
19.9&   	19.5&   	14.9&   	32\\
19.5&   	25.6&   	17&   	45.2\\
14.9&   	17&   	12.8&   	29\\
32&   	45.2&   	29.1&   	82.7\\
\end{array}} \right]\left[ {\begin{array}{c}
a\\
b\\
c\\
d\\
\end{array}} \right] = \left[ {\begin{array}{c}
21.2\\
32.2\\
19.2\\
58\\
\end{array}} \right]
\]

得出\begin{math}a = -0.31, b = 2.74, c = -1.21, d = -0.25\end{math},则该线性模型为:\begin{displaymath}f(x_{0}, x_{1}, x_{2}, x_{3}) = -0.31x_{0} + 2.74x_{1} - 1.21x_{2} - 0.25x_{3}\end{displaymath}

当这四个国家的经济增长率分别为1.2,0.3,-1.0和0.9时,代入可得美国的预期增长率为:

\begin{math}f(1.2, 0.3, -1.0, 0.9) = -0.31\times1.2 + 2.74\times0.3 - 1.21\times(-1.0) - 0.25\times0.9 
= 1.44\end{math}
\end{document}